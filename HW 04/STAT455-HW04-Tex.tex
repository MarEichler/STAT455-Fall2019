\documentclass[12pt, letterpaper]{article}

\input{C:/Users/MarEichler17/Documents/MYE-Documents/SCHOOL/Northwestern/preamble}
 



 %line spacing
 \usepackage{setspace}
 \setstretch{1.2}
 
 %header, pagestyle
 \geometry{pass, letterpaper}
 \pagestyle{fancy}
 \fancyhf{} % sets both header and footer to nothing
 \renewcommand{\headrulewidth}{0pt}
 \lhead{STAT 455 Fall 2019 \\ Homework 04}
 \rhead{Martha Eichlersmith \\ Page \thepage \text{ of} \pageref{LastPage}}
 \setlength{\headsep}{48pt}
 \renewcommand{\sectionmark}[1]{\gdef\currsection{\thesection \ #1}}
 \renewcommand{\subsectionmark}[1]{\markright{\currsection\ $\mid$ \thesubsection \  #1}}



\begin{document}

\subsection*{Problem {\#}3.3}  

\begin{wraptable}{l}{3in} 
\vspace{-12pt} 
\begin{tabular}{c l cc }
							&			& \multicolumn{2}{c}{Second Shot} \\
							&			& $S_2 = 1$	& $S_2 = 0$ 	\\ \cline{3-4} 
\multirow{2}{*}{First Shot} & $S_1 = 1$	& \multicolumn{1}{|c}{251}		& \multicolumn{1}{|c|}{34}\\ \cline{3-4} 
							& $S_1 = 0$ 	& \multicolumn{1}{|c}{48}		& \multicolumn{1}{|c|}{5} \\ \cline{3-4}
\end{tabular}
\vspace{-60pt} 
\end{wraptable} 

\quad \\[12pt]
$H_0: S_1 \independent S_2 \vs H_1:$ not $H_0$  
\\[36pt]
Data is not ordinal so a restricted alternative is not necessary.  

\begin{tabu}{X[l, 1] X[l, 1] X[l, 1] X[l, 4]}
\textbf{Statistics} & \textbf{Value} & \textbf{p-value} & \textbf{Conclusion} \\ \hline 
$X^2$ & 0.2727 & 0.6015 & Do not reject $H_0$, there is evidence that the first and second shot are independent \\ \hline
$G^2$ & 0.2858 & 0.5930 & Do not reject $H_0$, there is evidence that the first and second shot are independent \\ \hline 
\end{tabu}

\subsection*{Problem {\#}3.9(a)}

\begin{wraptable}{l}{3in} 
	\vspace{-12pt} 
	\caption{Standardized Pearson Residuals} 
	\centering
\begin{tabular}{lrr}
	\hline
	& Drugs & No Drugs \\ 
	\hline
	Schizophrenia & 7.87 & -7.87 \\ 
	Affective disorder & 1.60 & -1.60 \\ 
	Neurosis & -2.39 & 2.39 \\ 
	Personality disorder & -4.84 & 4.84 \\ 
	Special systems & -5.14 & 5.14 \\ 
	\hline
\end{tabular}
\end{wraptable}

The Pearson residuals are quite large which indicates that there is a lack of fit.  This is likely due to the different proportion of those who received drugs.  For Schizophrenia and Affective disorder, there are\textit{ more people who are} assigned drugs than those who are not assigned drugs.  Where as with Neurosis, Personality disorder and Special systems, there are \textit{more people who are not} assigned drugs than those who are assigned drugs.  This explains why the first two rows have standardized residuals with signs (+, -) and the last three rows have standardized residuals with signs (-, +).  

\subsection*{Problem {\#}3.12}
Gamma, $\gamma$: 0.3873 \\
95\% CI: (0.3156, 0.4591)

Gamma is 0.3873 which indicates that when attitudes disagree (i.e. counts that are not on the diagonal), the proportion of concordant attitudes towards abortions ($\uparrow$ school = $\uparrow$ approval) is larger than the proportion of discordant attitudes.  This means that there is greater approval of abortion when there is more schooling.  

\subsection*{Problem {\#}3.15}

\begin{tabular}{c l cc }
	&			& \multicolumn{2}{c}{Normalization} \\
	&			& Yes	& No	\\ \cline{3-4} 
	\multirow{2}{*}{Group} & Treatment	& \multicolumn{1}{|c}{7}	& \multicolumn{1}{|c|}{8}\\ \cline{3-4} 
	& Control							& \multicolumn{1}{|c}{0}	& \multicolumn{1}{|c|}{15} \\ \cline{3-4}
\end{tabular} \\

% latex table generated in R 3.6.1 by xtable 1.8-4 package
% Sun Nov 10 14:53:16 2019
\begin{table}[ht]
	\centering
	\begin{tabular}{rlll}
		\hline
		& V1 & CI Lower & CI Uppder \\ 
		\hline
		A & Woolf (Wald) & NaN & Inf \\ 
		B & Cornfield Exact & 1.9784 & Inf \\ 
		C(1) & Profile Likelihood & 0.0000 & NA \\ 
		C(2) & Profe Likelihood, counts+1 & 2.1242 & 286.7235 \\ 
		\hline
	\end{tabular}
\end{table}

Having a cell with a value of 0 makes analysis difficult.  

\subsection*{Problem {\#}3.31}
For a 2 $\times$2 table, consider $H_0: \pi_{11} = \theta^2, \quad \pi_{12} = \pi_{21} = \theta(1 - \theta),\quad  \pi_{22} = (1 - \theta)^2$.  
\subsubsection*{Problem {\#}3.31(a)} 
Show that the marginal distributions are identical and that independence holds.  

$\begin{array}{ccc}
\cline{1-2} 
\multicolumn{1}{|c}{\pi_{11}} & \multicolumn{1}{|c|}{\pi_{12}}  & \pi_{1+} \\ \cline{1-2}
\multicolumn{1}{|c}{\pi_{21}} & \multicolumn{1}{|c|}{\pi_{22}}  & \pi_{2+} \\ \cline{1-2}
\pi_{+1} & \pi_{+2} & 
\end{array}
\quad 
\stackrel{H_0}{=}  \quad 
\begin{array}{ccc}
\cline{1-2} 
\multicolumn{1}{|c}{\theta^2} & \multicolumn{1}{|c|}{\theta(1 - \theta)}  & \theta \\ \cline{1-2}
\multicolumn{1}{|c}{\theta(1 - \theta) } & \multicolumn{1}{|c|}{(1- \theta)^2}  & 1 - \theta \\ \cline{1-2}
\theta & 1 - \theta & 
\end{array}$ 

$ \left. 
\begin{array}{l }
\pi_{+1} = \pi_{+1} = \theta \\
\pi_{+2} = \pi_{+2} = 1 - \theta 
\end{array} \right\} \text{marginal distributions are the same}$

$ \left. 
\begin{array}{r c c c l }
\theta^2 &=& \pi_{11} = \pi_{1+} \pi_{+1} &=& \theta \cdot \theta = \theta^2 \\
\theta(1 - \theta) &=& \pi_{12} = \pi_{1+} \pi_{+2} &=& \theta \cdot (1 - \theta) \\
\theta(1 - \theta) &=& \pi_{21} = \pi_{2+} \pi_{+1} &=& (1 - \theta) \cdot \theta  \\
(1 - \theta)^2 &=& \pi_{22} = \pi_{2+} \pi_{+2} &=& (1 - \theta) \cdot (1 - \theta) = (1 - \theta)^2 
\end{array} \right\} \text{independence holds}$

\newpage 
\subsubsection*{Problem {\#}3.31(b)} 
\begin{align*}
f & \propto \pi_{11}^{n_{11}} \pi_{12}^{n_{12}} \pi_{21}^{n_{21}} \pi_{22}^{n_{22}}
\\
& \stackrel{H_0}{=} (\theta^2)^{n_{11}} \cdot (\theta(1 - \theta))^{n_{12}+n_{21}} \cdot ((1 - \theta)^2)^{n_{22}}
\\
L & =  n_{11} \log\left[ \theta^2 \right] + (n_{12} + n_{21}) \log \left[ \theta(1 - \theta) \right] + n_{22} \log \left[ (1 - \theta)^2 \right] 
\\
\frac{ \partial L}{ \partial \theta} & = \frac{2 n_{11}}{\theta} + \frac{n_{12} + n_{21}}{\theta} - \frac{n_{12} + n_{21}}{(1 - \theta)} - \frac{ 2 n_{22} }{(1 - \theta) }
\\[0.5ex]
& = \frac{ 2 n_{11} + n_{12} + n_{21}}{\theta} - \frac{2 n_{22} + n_{12} + n_{21}}{1 - \theta} \set 0 
\\[0.5ex]
\implies \hat{\theta} & = \frac{ 2 n_{11} + n_{12} + n_{21} }{2(n_{11} + n_{12} + n_{21} + n_{22}  )} 
\qquad \qquad  \begin{tiny}
\begin{array}{r c l}
n &=& n_{11} + n_{12} + n_{21} + n_{22} \\
n_{+1} &=&  n_{11} + n_{21} \\
n_{1+} &=& n_{11} + n_{12}
\end{array}
\end{tiny}
\\[1ex]
& = \frac{ n_{1+} + n_{+1} }{2 n} = \frac{1}{2} \left( \frac{n_{1+}}{n} \cdot \frac{n_{+1}}{n} \right)  
\\[0.5ex]
\hat{\theta} &= \frac{ p_{1+} + p_{+1} }{2}
\end{align*}

\subsubsection*{Problem {\#}3.31(c)}

Calculate the expected frequencies: 

$
\arraycolsep=1.5pt \def\arraystretch{1.2}
\begin{array}{r c l c l}
\hat{\mu}_{11} & = & n \cdot \hat{\pi}_{11} &=& n \cdot \hat{\theta}^2 \\
\hat{\mu}_{12} & = &n \cdot \hat{\pi}_{12} &=& n \cdot \hat{\theta}(1 - \hat{\theta}) \\ 
\hat{\mu}_{21} & = &n \cdot \hat{\pi}_{21} &=& n \cdot \hat{\theta}(1 - \hat{\theta}) \\
\hat{\mu}_{22} & = &n \cdot \hat{\pi}_{22} &=& n \cdot (1 - \hat{\theta})^2
\end{array}$ 

Using the expected frequencies obtain Pearson's $X^2 = \sum \frac{ \left( n_{ij} - \hat{\mu}_{ij} \right)^2 }{ \hat{\mu}_{ij} } $ and compare to a $\chi^2$ distribution with $df= 2$ ($df = \text{dim}(H_0 \cup H_1) - \text{dim}(H_0) = 3 - 1 = 2$).  Note that when testing independence for a 2$\times$2 table $df = 1$.   


\subsubsection*{Problem {\#}3.31(c)}

\tabulinesep=1mm
\begin{tabu}{X[l, .7] X[l, 1] X[l, .3] X[l, 1] X[l, 5]}  
\textbf{$H_0$} 	& $X^2$ 	& $df$  & \textbf{p-value}	& \textbf{Conclusion} \\ \hline 
	indep.		& 0.27274 	& 1 	& 0.6015 			& Do not reject $H_0$, there is evidence that the first and second shot are independent  \\ \hline
	iid			& 0.27274 	& 2 	& 0.8725 			& Do not reject $H_0$, there is evidence that the first and second shot are iid  \\\hline  
\end{tabu}

It is plausible that the free throws are independent and identically distributed.  









\end{document}



%  \text{\textcolor{red}{$$}}
%  \text{\textcolor{blue}{$$}}
%  \text{\textcolor{Green}{$$}}