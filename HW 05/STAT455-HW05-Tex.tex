\documentclass[12pt, letterpaper]{article}


\input{C:/Users/MarEichler17/Documents/MYE-Documents/SCHOOL/Northwestern/preamble}
 
 %line spacing
 \usepackage{setspace}
 \setstretch{1.2}
 
 %header, pagestyle
 %\geometry{pass, letterpaper}
 \geometry{letterpaper, left=.75in, right=.75in, top=1.2in, bottom=1in}
 \pagestyle{fancy}
 \fancyhf{} % sets both header and footer to nothing
 \renewcommand{\headrulewidth}{0pt}
 \lhead{STAT 455 Fall 2019 \\ Homework 05}
 \rhead{Martha Eichlersmith \\ Page \thepage \text{ of} \pageref{LastPage}}
 \setlength{\headsep}{48pt}
 \renewcommand{\sectionmark}[1]{\gdef\currsection{\thesection \ #1}}
 \renewcommand{\subsectionmark}[1]{\markright{\currsection\ $\mid$ \thesubsection \  #1}}

\begin{document}

\section{Common Probability Model} 
This model assumes that each game has the same free-throw conversion rate, i.e. : $\pi_i = \alpha$.    

\subsection{Estimate $\alpha$} 
We run a regression of the free throws (binary data, 0 or 1) versus a constant to get an estimation for $\alpha$ (the intercept).  
% latex table generated in R 3.6.1 by xtable 1.8-4 package
% Wed Nov 20 15:20:38 2019
\begin{table}[ht]
	\centering
	\caption{Common Probability Model -  $\alpha$} 
	\begin{tabular}{clr}
		\hline
		& Statistic & Values \\ 
		\hline
		1 & $\hat{\alpha}$ & 0.456 \\ 
		2 & Standard Error & 0.029 \\ 
		3 & Confidence Interval & ( 0.400,0.513 ) \\ 
		\hline
	\end{tabular}
\end{table}

Shaq's estimated probability of making a free throw is 0.456 [95\% CI (0.400, 0.513)] for each of the 23 games.  

\subsection{Global Lack of Fit - Chi-Squared Approximation} 
One way to check the global goodness of fit of a model is with chi-squared approximation.  However, prior to using the chi-squared approximation, we have to check if the assumption is met, i.e. we have to check if $n^* \gg \text{dim}(H_0 \cup H_1)$.  In this case, $n^* = 296$, which is the total number of attempts.  Also, $\text{dim}(H_0 \cup H_1)$ = 23, the number of games or observations.  $296 \gg 23$ so we can proceed with the chi-squared approximation.  

Pearson's X-squared test statistic is 35.511, which corresponds to a p-value of 0.034 given 22 degrees of freedom. Note that degrees fo freedom is $\text{dim}(H_0 \cup H_1)  - \text{dim}(H_0) = 23-1- 22$ degrees of freedom. Given that the p-value is small, we reject the null hypothesis that the model of constant probability across the 23 games hold.  In other words, we have evidence that there is a lack of fit.   

\subsection{Local Lack of Fit - Standardized Residuals} 
Another way to look at goodness of fit, is to check standardized residuals for local lack of fit.  Generally, if a standardized residual is larger than 2 in absolute value, that is considered an indication of local lack of fit.  

We  have one standardized residual that is larger than 2, which demonstrates a local lack of fit for the 14th Game (residual=3.327).  Besides the residual for the 14th game, there are no other outliers.  The significant lack of fit for the 14th game is due to having such a different proportion than the other games.  In the 14th game Shaq shot 9/9 (100\%), all of the other proportions are between 0.167 and 0.800. 


\subsection{Game 14 - Outlier causing lack of fit?}
If a chi-squared approximation is used to analysis goodness of fit on the data \textit{without Game 14}, the result is a $X^2 = 24.618$ on 21 degrees of freedom with a p-value of 0.264.  We would fail to reject the null hypothesis that the model of common probability holds for the 23 games.  In addition, there are no standardized residuals that have an absolute value greater than 1.693, suggest there is no local lack of fit.  

\section{Common Probability Model with Over-dispersion} 
If our previous assumption of independent trials (one free throw is independent of another) is incorrect, then our previous goodness of fit tests and our estimates of $\alpha$ and its standard error have questionable validity since those values were computed under the assumption that there was independence between trials.   

Is it feasible that the trials (i.e. free throws) are not independent? It is possible that a player within a game will be doing well, having successful free throws, and will build off that momentum and confidence to continue playing well.  In sports terms, this means getting a ``hot hand" and means a players is able to have successful shots multiple times in a row.  If this situation happens, there is positive within-game correlation.  When there is a positive within-game correlation, this will lead to over-dispersion to what is expected in the binomial model that assumes independence.  

In our example, look at Game 14 where Shaq shot 9/9 on the free throw line.  His ``hot hand" in this game leads to a more success  with likely dependent trials than expected under our common probability model.  

The Common Probability model rejection (using the independence assumption) has two possible reasons: (1) the Common Probability assumption is not true or (2) the Common Probability assumption could be true, but the interdependent assumption is not. 

We will look at the second reason, keeping in mind Game 14 when Shaq went 9/9, much higher than any other game.  We expect over-dispersion if we assume our ``hot hand'' theory is true which leads to positive within-game correlation.  

We need to assume the model includes an unknown dispersion parameter, this will update our results.    

\subsection{Estimate Dispersion and Scale} 
For a binomial model, we estimate the dispersion $\hat{\phi} = X^2 / df$ where $X^2$ is the Pearson X-squared statistic  and $df$ is the degrees of freedom used in the chi-squared approximation.  We have $\hat{\phi} = \nicefrac{35.511}{22} = 1.614$.  

So our estimate for dispersion, $\hat{\phi}$ is 1.614 and our estimate for scale is $\sqrt{ \hat{\phi}}$ is 1.270.  This suggest that there is over-dispersion.  


\subsection{Estimate $\alpha$, assuming $\phi \neq 1$} 

We are able to use the dispersion and scale to adjust the approximate standard error for $\alpha$ and thus the confidence interval.  The common probability model, assuming $\phi \neq 1$,  approximate standard error is related to the common probability model, assuming $\phi =1$, approximate standard error in the flowing manner:  
\begin{align*}
\se{\alpha}_{\phi \neq 1}  & = \sqrt{\hat{\phi}} \cdot \se{\alpha}_{\phi = 1} 
\end{align*}


% latex table generated in R 3.6.1 by xtable 1.8-4 package
% Wed Nov 20 17:41:56 2019
\begin{table}[ht]
	\centering
	\caption{Common Probability Model assuming $\phi \neq 1$ - $\alpha$} 
	\begin{tabular}{llr}
		\hline
		& Statistic & Values \\ 
		\hline
		1 & $\hat{\alpha}$ & 0.456 \\ 
		2 & Standard Error of $\alpha$ & 0.037 \\ 
		3 & Confidence Interval for $\alpha$ & ( 0.384,0.528 ) \\ 
		\hline
	\end{tabular}
\end{table}

From the results above, given the over-dispersion, the standard error increases and the confidence interval widens.  

\section{Conclusion}
The common probability model, that chance of success in free throws is the same for all 23 games, reasonably hold.  With the exception of the large outlier from the 14th game, the observed proportions are in line with those predicted using the common probability model.  This suggests that Shaq's shooting percentage variability can be explained by randomness.  There is no strong case that the success probability varies from game to game.  There is some over-dispersion, likely due to within-game correlation among the trials.  Taking into account over-dispersion, Shaq's estimated probability of making a free throw is 0.456 [95\% CI (0.384, 0.528)] for each of the 23 games.  



\end{document}



%  \text{\textcolor{red}{$$}}
%  \text{\textcolor{blue}{$$}}
%  \text{\textcolor{Green}{$$}}