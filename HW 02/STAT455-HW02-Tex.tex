\documentclass[12pt, letterpaper]{article}

\input{C:/Users/MarEichler17/Documents/MYE-Documents/SCHOOL/Northwestern/preamble}
 



 %line spacing
 \usepackage{setspace}
 \setstretch{1.2}
 
 %header, pagestyle
 \geometry{pass, letterpaper}
 \pagestyle{fancy}
 \fancyhf{} % sets both header and footer to nothing
 \renewcommand{\headrulewidth}{0pt}
 \lhead{STAT 455 Fall 2019 \\ Homework 02}
 \rhead{Martha Eichlersmith \\ Page \thepage \text{ of} \pageref{LastPage}}
 \setlength{\headsep}{48pt}
 \renewcommand{\sectionmark}[1]{\gdef\currsection{\thesection \ #1}}
 \renewcommand{\subsectionmark}[1]{\markright{\currsection\ $\mid$ \thesubsection \  #1}}



\begin{document}

\subsubsection*{Problem {\#}1.a}  
\begin{multicols}{2}
\begin{align*}
\omit\rlap{\textbf{Restricted:}} \\
p(\alpha \mid \yb) & = y_1^\alpha y_2^\alpha y_3^{1 - 2 \alpha} 
\\
L(\alpha \mid \yb) & = (y_1 + y_2) \log \alpha + y_3 \log (1 - 2 \alpha)
\\
s(\alpha \mid \yb) & = \frac{ y_1 + y_2}{\alpha} - \frac{2 y_3}{1 - 2\alpha }
\\[0.5ex]
B(\alpha ) & = \frac{ 2n}{\alpha(1 - 2 \alpha)} 
\\[0.5ex]
\alpha_0 & = \nicefrac{1}{3} 
\\
s(\alpha_0 \mid \yb) & = \frac{y_1 + y_2}{\alpha_0} - \frac{2 y_3}{1 - 2 \alpha_0} 
\\[0.5ex]
&= \frac{ y_1 + y_2 - 2\left(n - y_1 - y_2\right)}{\nicefrac{1}{3}} 
\\[0.5ex]
& = \frac{ y_1 + y_2 - 2n + 2y_1 + 2y_2}{\nicefrac{1}{3}} 
\\[0.5ex]
& = \frac{ 3\left( y_1 + y_2 - \frac{2}{3}n\right)}{ \nicefrac{1}{3}}
\\[0.5ex]
& = 9 \cdot \left(y_1 + y_2 - \tfrac{2}{3} n\right)
\\[0.5ex]
B(\alpha_0) & = \frac{2n}{\alpha_0 (1 - 2 \alpha_0)} 
\\[0.5ex]
& = \frac{ 2 n}{ \frac{1}{3} \cdot \frac{1}{3} } = \frac{2n}{\nicefrac{1}{9}} = 9 \cdot 2n 
\\[0.5ex]
& = 18 n 
\\[1ex]
S^2_R & = \frac{s(\alpha_0 \mid \yb)^2 }{B(\alpha_0)} 
\\[0.5ex]
&= \frac{( 9 \cdot \left(y_1 + y_2 - \frac{2}{3} n\right))^2  }{18 n  } 
\\[0.5ex]
& =  \frac{ 9 \cdot 9 \cdot \left(y_1 + y_2 - \tfrac{2}{3} n\right)^2 }{ 9 \cdot 2 \cdot n }
\\[0.5ex]
& = \frac{ \left[ y_1 + y_2 - \left( \nicefrac{2}{3} \right) n \right]^2 }{ \left( \nicefrac{2}{9} \right)n} 
\end{align*}

\columnbreak 
\begin{align*}
\omit\rlap{\textbf{Unrestricted:}} \\
\hat{m}_{0, i} & = n \cdot \nicefrac{1}{3} = \nicefrac{n}{3} 
\\
S^2_U & = \sum_{i=1}^3 \frac{ \left( y_i - \hat{m}_{0, i} \right)^2}{\hat{m}_{0, i} }
\\[0.5ex]
& = \sum_{i=1}^3 \frac{ \left( y_i - \nicefrac{n}{3} \right)^2}{\nicefrac{n}{3}  }
\\[0.5ex]
& = 
\frac{ \left( y_1 - \nicefrac{n}{3} \right)^2}{\nicefrac{n}{3}  }
+
\frac{ \left( y_2 - \nicefrac{n}{3} \right)^2}{\nicefrac{n}{3}  }
+
\frac{ \left( y_3 - \nicefrac{n}{3} \right)^2}{\nicefrac{n}{3}  }
\end{align*}

\end{multicols}

\newpage 

\subsubsection*{Problem {\#}1.b}  
% latex table generated in R 3.6.1 by xtable 1.8-4 package
% Wed Oct 16 13:46:34 2019
\begin{table}[ht]
	\centering
	\begin{tabular}{rllllllll}
		\hline
		& Sample.Size & pi.T1 & pi.T2 & pi.T3 & P.R & aP.R & P.U & aP.U \\ 
		\hline
		1 &  75 & 0.3333333 & 0.3333333 & 0.3333333 & 0.0373 & 0.0500 & 0.0508 & 0.0500 \\ 
		2 &  75 & 0.2500000 & 0.2500000 & 0.5000000 & 0.8242 & 0.8647 & 0.7795 & 0.7884 \\ 
		3 &  75 & 0.1666667 & 0.5000000 & 0.3333333 & 0.0378 & 0.0500 & 0.9216 & 0.8962 \\ 
		4 &  75 & 0.2000000 & 0.3000000 & 0.5000000 & 0.8238 & 0.8647 & 0.8397 & 0.8349 \\ 
		5 & 250 & 0.3333333 & 0.3333333 & 0.3333333 & 0.0519 & 0.0500 & 0.0467 & 0.0500 \\ 
		6 & 250 & 0.3000000 & 0.3000000 & 0.4000000 & 0.6256 & 0.6088 & 0.4902 & 0.5037 \\ 
		7 & 250 & 0.2200000 & 0.4467000 & 0.3333000 & 0.0542 & 0.0500 & 0.9868 & 0.9819 \\ 
		8 & 250 & 0.2500000 & 0.3000000 & 0.4500000 & 0.9727 & 0.9746 & 0.9543 & 0.9594 \\ 
		9 & 250 & 0.2200000 & 0.4000000 & 0.3800000 & 0.3721 & 0.3467 & 0.9598 & 0.9381 \\ 
		\hline
	\end{tabular}
\end{table}
The power tends to increase when the true probabilities align with the alternative and when there is an increasing in sample size.  

\subsubsection*{Problem {\#}1.c} 
% latex table generated in R 3.6.1 by xtable 1.8-4 package
% Wed Oct 16 13:20:38 2019
\begin{table}[ht]
	\centering
	\begin{tabular}{rlllll}
		\hline
		& pi.T1 & pi.T2 & pi.T3 & n.R & n.U \\ 
		\hline
		1 & 0.3333333 & 0.3333333 & 0.3333333 & Inf & Inf \\ 
		2 & 0.2500000 & 0.2500000 & 0.5000000 & 63 & 78 \\ 
		3 & 0.1666667 & 0.5000000 & 0.3333333 & Inf & 58 \\ 
		4 & 0.2000000 & 0.3000000 & 0.5000000 & 63 & 69 \\ 
		5 & 0.3000000 & 0.3000000 & 0.4000000 & 393 & 482 \\ 
		6 & 0.2200000 & 0.4467000 & 0.3333000 & 1569772001 & 125 \\ 
		7 & 0.2500000 & 0.3000000 & 0.4500000 & 129 & 149 \\ 
		8 & 0.2200000 & 0.4000000 & 0.3800000 & 801 & 165 \\ 
		\hline
	\end{tabular}
\end{table}

It makes sense that when the true probabilities are equal that is no sample possible to get 80\% power to detect the differences.  It also makes sense that when the power calculated in part b is lower than 80\% a larger sample than in part b is needed to achieve that power.  When the calculated power is greater than 80\%, then a smaller sample is needed.   



\end{document}



%  \text{\textcolor{red}{$$}}
%  \text{\textcolor{blue}{$$}}
%  \text{\textcolor{Green}{$$}}